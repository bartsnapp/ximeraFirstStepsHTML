\documentclass{ximera}
\usepackage{tikz}
\usepackage{color}
\usepackage{amsmath,amssymb,amsfonts}
%\usepackage{helvet}
%\renewcommand{\familydefault}{\sfdefault}
\author{Jeffrey Kuan}
\input{../preamble} %% Loads the graphics path
\title{The FRT Construction}
\license{CC: 0}
\begin{document}

\begin{abstract}
   The Yang--Baxter equation is cool. 
\end{abstract}
\maketitle
%\part{Introduction}
%\chapterstyle
%    \activity{basics/basicWorksheet}
%    \sectionstyle
%    \activity{basics/exercises/someExercises}

%    \chapterstyle
%    \activity{basics/graphicsInteractives}

Accessibility statement: \href{https://ximera.osu.edu/firststeps24html/aFirstStepInXimera/basics/FRT_Construction_WCAG}{A WCAG2.1AA compliant version of these notes} will eventually available, once I finish writing them. I tested it out using NVDA, a screen reader.
It works best if you uncheck ``graphic'' and ``clickable'' in the settings. I am also testing out dyslexia--friendly 
fonts right now. 
You can also \href{https://ximera.osu.edu/firststeps24html/aFirstStepInXimera/basics/FRT_Construction.tex}{download the TeX source file.}

\section{About this webpage}
These lecture notes were created with \href{https://ximera.osu.edu/}{Ximera}, an 
interactive textbook platform hosted by Ohio State University. The Ximera Project is funded 2024-2026 (with no other external funding) by a 
\$2,125,000 \href{https://www.ed.gov/grants-and-programs/grants-higher-education/improvement-postsecondary-education/open-textbooks-pilot-program}{Open Textbooks Pilot Program} grant from the federal Department of Education.

These notes have not yet been peer--reviewed. However, it is updated with all three lectures.
To load the most updated version, click the orange ``update'' button
at the top of the page. If it is not there, then you are reading the most up-to-date version. The button looks like this:

\includegraphics{UpdateButton.png}

Funding was provided by \href{https://www.lms.ac.uk/grants/visits-uk-scheme-2}{The London Mathematical Society} and 
Lancaster University, University of Bristol, University of Edinburgh, Warwick University, Queen Mary University of London, and Ximera. 






\begin{thebibliography}{10}

\bibitem{FRT88} L. D. Faddeev, N. Yu. Reshetikhin, L. A. Takhtajan, Quantization
of Lie groups and Lie algebras, Algebraic analysis, Vol. I, Academic
Press, Boston, MA, (1988), 129-139.

\end{thebibliography}

\end{document}