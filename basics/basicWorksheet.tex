\documentclass{ximera}

\author{Jeffrey Kuan}
\input{../preamble} %% Loads the graphics path
\title{Introducton to Quantum Groups and Yang-Baxter Equation For Probabilists}
\license{CC: 0}
\begin{document}
%\begin{abstract}
%    The Yang--Baxter equation is cool. 
%\end{abstract}
%\maketitle
%\part{Introduction}
%\chapterstyle
%    \activity{basics/basicWorksheet}
%    \sectionstyle
%    \activity{basics/exercises/someExercises}

%    \chapterstyle
%    \activity{basics/graphicsInteractives}
\section{Introduction}

Broadly speaking, \textbf{integrable probability} is a branch of probability theory which studies 
models which have exact solutions. For this reason, these models are often called \textbf{exactly solvable.}
From these exact solutions, one can derive precise \textbf{universal} asymptotics, such as the 
famed Tracy--Widom distribution. The origin of these solutions are usually due to algebraic symmetries 
underlying the model. In these set of lecture notes, we will introduce the relevant algebraic background
with a probabilistic researcher as the target audience. 

For historical context, the terminology originates in integrable systems from Hamiltonian mechanics.
In Hamiltonian mechanics, the phase space is represented as a smooth manifold with even dimension $2n,$
with coordinates denoted $q_1,\ldots,q_n,p_1,\ldots,p_n$ for the momentumm and position. An integrable
system is a system with $n$ conserved quantities, and the Liouville--Arnold theorem states that the 
equations of motion can be solved in quadratures. For an explicit example, the harmonic oscillator in 
one dimension (imagine an object attached to a frictionless spring) is integrable, and the conserved quantity
is the total energy. In contrast, the three--body problem is not integrable, and its solutions are 
notoriously non--exact. 

During the 1980s, many Soviet mathematicians and physicsts introduced quantum mechanics into integrable
systems. In quantum mechanics, quantities such as position, momentum and energy became operators which 
generally do not commute; additionally, the values of these quantities only take discrete, ``quantized''
values. In this context, the concept of ``conserved quantities'' becomes ``commuting operators,'' and 
values of the quantities are eigenvalues of eigenstates. At the time, their approaches were called the
``quantum inverse scattering method.'' Since then, the method has been generalized to abstract 
algebraic objects, such as Hopf algebras. 

These notes will introduce one such algebraic object, known as \textbf{quantum groups,} with a particular
focus on the Yang--Baxter equation. The exposition will use probability and mathematical physics as a
motivation. The goal is for a reader with a probability background to be able to read contemporary (as of
2024) research papers in integrable probability. 

\textbf{Acknolwedgements.} The author was supported by the London Mathematical Society. 

\section{ASEP, XXZ, and $U_q(sl_2)$}

To begin to motivate the notes, we first introduce the asymmetric simple exclusion process (ASEP) and its relationship to 
the Heisenberg XXZ model and the quantum group $U_q(sl_2).$ 

\subsection{ASEP}
In ASEP, particles randomly jump on a lattice, which we assume to be
one--dimensional. At most one particle may occupy a site, and jumps to occupied sites are blocked (hence the term ``exclusion'').
Jumps are nearest neighbour (hence the term ``simple''). If the jumps are continuous--time exponential 
clocks with left rates $\alpha$ and right rates $\beta,$ then let $q=\sqrt{\beta/\alpha$} denote the 
asymmetry parameter. If $q\neq 1,$ then the model is asymmetric (sometimes partially asymmetric), 
while for $q=1$ the model is symmetric. For $q=0$ or $q=\infty$ the model is called totally asymmetric.
The symmetric exclusion process (without the word simple) can be defined more generally on an arbritrary graph. In princple, so can 
the asymmetric exclusion process, although this is not as well studied.  

\includegraphics{ASEPScreenshot.png}

The generator of the simple exclusion process can be explicitly written. In the most elementary case 
where there are two lattice sites, then there are four possible configurations. Associate to each
particle the vector $[1\ 0]$ and to each hole (i.e. a non--particle) the vector $[0 \ 1].$ Tensoring
the vectors together, we can associate to each configuration a canonical basis element of the four--dimensional
vector space $\mathbb{C}\otimes \mathbb{C}.$ The field here is chosen to be the complex numbers because
it is algebraically closed, although of course probabilities are real numbers. 

\includegraphics{ASEPConfigurations.png}

With that set up, the generator is then a $4\times 4$ matrix
$$
\alpha\left(
    \begin{array}{cccc}
       0 & 0 & 0 & 0 \\
       0 & -1 & 1 & 0 \\
       0 & q^2 & -q^2 & 0 \\
       0 & 0 & 0 & 0 
    \end{array}
\right).
$$
The constant $\alpha$ can be viewed as a time rescaling, so it can be removed without loss of generality. 

In a more general setting, where there are $N$ lattice sites, the generator can be defined from the
above \(4\times 4\) matrix, which we now denote \(\mathcal{L}.\) The generator is now a \(2^N \otimes 2^N\) and
acts on  \((\mathbb{C}^2)^{\otimes N}.\) Define \(\mathcal{L}_{i,i+1}\) by 
\[
\mathcal{L}_{i,i+1} = (\mathrm{Id}_2)^{i-1} \otimes \mathcal{L} \otimes (\mathrm{Id}_2)^{N-1}.
\]
With this notation, the generator is
\[
\sum_{i=1}^{N-1} \mathcal{L}_{i,i+1}.
\]
Similar notation using subscripts will be used throughout these notes.

\subsubsection{Yang--Baxter Equation}

Using the same notation, we say that a matrix \(R\) solves the \underline{braided Yang--Baxter Equation} if:
\[
R_{12}R_{23}R_{12} = R_{23}R_{12}R_{23}.
\]
Somewhat confusingly, this equation is also sometimes called the braid equation or just the Yang--Baxter equation.
An equivalent formulation is
\[
R_{12}R_{13}R_{23} = R_{23}R_{13}R_{12}
\]
(exercise left to the reader). Given an arbitrary matrix \(R,\) one can check that it satisfies the YBE
through direct computation. However, it maybe more helpful to consider some simple examples first. 

If \(P: V \otimes V \rightarrow V \otimes V\) denotes the permutation operator which sends
\(u \otimes v\) to \(v \otimes u,\) then the braided Yang--Baxter equation becomes an identity of transpositions:
\[
(1\ 2) \circ (2 \ 3 ) \circ (1 \2 ) = (2 \ 3) \circ (1 \ 2) \circ (2 \ 3).
\]
This identity holds, since both sides equal the permutation \((1\ 3).\) An even more simple solution is 
the identity matrix. 

A probabilist may remark that a permutation matrix and an identity matrix are both examples of a stochastic matrix,
albeit somewhat trivial examples. It may then be natural to try to find more general stochastic matrices
which solve the Yang--Baxter equation. The ``simplest'' generalization occurs when \(V\) is two--dimensional,
with basis \(e_1,e_2.\) Using exclusion processes as a prototype, we can define the operator $R_{\alpha\beta}$
by
\[
R_{\alpha\beta}(e_1 \otimes e_1) = e_1 \otimes e_1, \quad R_{\alpha\beta}(e_2\otimes e_2) = e_2 \otimes e_2
\]
and
\begin{align*}
R_{\alpha\beta}(e_1 \otimes e_2) &= (1-\alpha)e_1 \otimes e_2 + \alpha e_2 \otimes e_1,\\
R_{\alpha\beta}(e_2 \otimes e_1) &= \beta e_1\otimes e_2 + (1-\beta)e_2 \otimes e_1.
\end{align*}
Inserting this matrix into the Yang--Baxter equation and evaluating at \(e_1 \otimes e_1 \otimes e_2,\)
one finds that \(\alpha(1-\beta)=0\) is a necessary requirement for a solution to YBE. Evaluating
\( R_{\alpha,1}\) and \(R_{0,\beta}\) at \(e_1\otimes e_2 \otimes e_1\) and \(e_2 \otimes e_1 \otimes e_1\)
yields no aditional equations. (Exercise left to the reader)

One could also evaluate YBE at the three vectors \(e_2 \otimes e_1 \otimes e_1,e_1\otimes e_2 \otimes e_1\)
and \(e_1 \otimes e_1 \otimes e_2,\) but instead we use another symmetry of ASEP. This is called
\underline{particle--hole involution.} In words, in an ASEP with drift to the right, the holes evolve
as an ASEP with drift to the left. Symbolically, we define an involution \(T\) of \(V\) which
switches \(e_1\) and \(e_2.\) Then 
\[
T^{\otimes 2} R_{\alpha,\beta}T^{\otimes 2} = R_{\beta,\alpha}, \quad \quad T^{\otimes 2} R_{\alpha,\beta} = R_{1-\alpha,1-\beta}, \quad \quad R_{\alpha,\beta}T^{\otimes 2} = R_{1-\beta,1-\alpha}.
\]
A mildly helpful observation here is that \(T\) is stochastic, so its composition with other stochastic
matrices is also stochastic. Using these identities, one immediately verifies YBE for 
\(R_{\alpha,0}\) and \(R_{1,\beta}.\) Furthermore, these two solutions are related up to the particle--hole involution.



\subsection{XXZ model}
We now turn our attention to a related model, called the quantum Heisenberg model. These were introduced
by Werner Heisenberg to incorporate quantum mechanics into magnetism. Each lattice site has a microscopic
magnetic dipole, which can either be up or down. If there are \(N\) lattice sites, 
then the Hamiltonian acts on the \(2^N\)--dimensional vector space \(\mathbb{C}^2)^{\otimes N}.\)

\section{Yang--Baxter Equation}
State equation. Relationship to braids and permutationa and identities. 

Solution for 4 by 4 with ASEP. 

State that adding a constant keeps eigenvectors same and only changes eigenvalues



\(U_q(sl_2).\)

parameter dependent YBE and affine lie algebras

\section{QG and then Hopf algebras}
What is coproduct? 

RTT construction?

\section{Solutions to YBE, explicitly (less algebraic)}

Fusion

Orthogonal polynomial vertex weights?

\section{Research?}

\end{document}